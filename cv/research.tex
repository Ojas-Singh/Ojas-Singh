\datedsubsection{\scriptsize{Jan 2021 - Current}}
{%
\textbf{Dr. P. Balanarayan, IISER Mohali}}
	{%
		\textbf{Time-Dependent Configuration Interaction} ~(\emph{Masters Thesis})}
		{%
		I have developed a fast CI code for TDCI using BitArray and concurrent data structures in pure Rust. We have successfully leveraged Rust's rich type system to go beyond LLVM's optimizations, making our implementation extremely efficient. I am currently working on Time Propagation using the (t, t′) method.		
		  \begin{multicols}{2}\raggedright % adjust cols as needed
			\begin{itemize}
				\item[\circ] Concurrent Programming / Compiler Optimizations
				\item[\circ] Correlated Methods
				% \item[\circ] Sparse Matrix Algebra
			  \end{itemize}
		  \columnbreak
		  
		  \begin{itemize}
			\item[\circ] Sparse Matrix Algebra
			\item[\circ] BitArray Manipulation
			% \item[\circ] Quantum Dynamics
		  \end{itemize}
		  \columnbreak
		  
		\end{multicols}
		}




\datedsubsection{\scriptsize{Feb 2021 - June 2021}}
{%
\textbf{Dr. S Rakshit, IISER Mohali	}}
	{%
		\textbf{High-Performance Parallel Algorithm for Magnetic Tweezer}~for real-time monitoring of protein folding and unfolding}
		{%
		I have designed and developed a high-performance parallel python program for real-time Image Acquisition and Processing. We were able to analyze fast magnetic bead movement at the millisecond temporal resolution and nanometer spatial resolution. This implementation can process 3000+ frames per second which is three times faster than industry ­leading software.
		\begin{multicols}{2}\raggedright % adjust cols as needed
			\begin{itemize}
				\item[\circ] High-Performance Python Programming
				% \item[\circ] Protein Folding / Unfolding  
				% \item[\circ] 
			  \end{itemize}
		  \columnbreak
		  
		  \begin{itemize}
			\item[\circ] Image Acquisition / Analysis and Signal Processing
			% \item[\circ] Single Molecule Study / Magnetic Tweezer
			% \item[\circ] Protein Folding/Unfolding 
		  \end{itemize}
		  \columnbreak
		  
		\end{multicols}
		}


\datedsubsection{\scriptsize{May 2019 - Dec 2020}}
{%
\textbf{Dr. P. Balanarayan, IISER Mohali}	}
	{%
		\textbf{Spectral Clustering Based Fragmentation Approach for Ab-initio Quantum Calculation}~of large molecules}
		{%
		A Spectral-Clustering-based fragmentation scheme for estimating the electronic energy of a large molecule using ab­-initio methods is devised. The method exploits salient properties of graphs to predict the best possible fragments and overlaps heuristically. In the tests performed, energy estimates obtained using this method show an excellent agreement with those obtained via the actual computation of the complete molecule. The accuracy of the results obtained deploying this method allows the quantum calculation of large molecules.
		\begin{multicols}{2}\raggedright % adjust cols as needed
			\begin{itemize}
				\item[\circ] Spectral Graph Partitioning Algorithm
				\item[\circ] Python Programming Language
				% \item[\circ] Molecule Visualization 
				\end{itemize}
			\columnbreak
			
			\begin{itemize}
			\item[\circ] Gaussian Software Package
			% \item[\circ] 
			\item[\circ] Scripting for Automated Computation
			\end{itemize}
			\columnbreak
			
		\end{multicols}
		}


\datedsubsection{\scriptsize{April 2017 - May 2017}}
{%
\textbf{Prof. Sudeshna Sinha, IISER Mohali	}}
	{%
		\textbf{Non-linear Dynamics and Chaos Theory }~_}
		{%
		Studied and analysed bifurcations in simple logistic maps, Lorenz systems and three dimensional chaos using
		C++ and Python.
		% \begin{multicols}{2}\raggedright % adjust cols as needed
		% 	\begin{itemize}
		% 		\item[\circ] Multi-Processing Programming
		% 		\item[\circ] High Performance Python 
		% 		\item[\circ] Image Acquisition
		% 		\end{itemize}
		% 	\columnbreak
			
		% 	\begin{itemize}
		% 	\item[\circ] FFT
		% 	\item[\circ] Magnetic Twizzer
		% 	\item[\circ] Protien Folding/Unfolding 
		% 	\end{itemize}
		% 	\columnbreak
			
		% \end{multicols}
		}